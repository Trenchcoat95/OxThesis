%%%%%%%%%%%%%%%%%%%%%%%%%%%%%%%%%%%%%%%%%%%%%%%%%%%%%%%%%%%%%%%
%% OXFORD THESIS TEMPLATE

% Use this template to produce a standard thesis that meets the Oxford University requirements for DPhil submission
%
% Originally by Keith A. Gillow (gillow@maths.ox.ac.uk), 1997
% Modified by Sam Evans (sam@samuelevansresearch.org), 2007
% Modified by John McManigle (john@oxfordechoes.com), 2015
%
% This version Copyright (c) 2015-2023 John McManigle
%
% Broad permissions are granted to use, modify, and distribute this software
% as specified in the MIT License included in this distribution's LICENSE file.
%

% I've (John) tried to comment this file extensively, so read through it to see how to use the various options.  Remember
% that in LaTeX, any line starting with a % is NOT executed.  Several places below, you have a choice of which line to use
% out of multiple options (eg draft vs final, for PDF vs for binding, etc.)  When you pick one, add a % to the beginning of
% the lines you don't want.


%%%%% CHOOSE PAGE LAYOUT
% The most common choices should be below.  You can also do other things, like replacing "a4paper" with "letterpaper", etc.

% This one will format for two-sided binding (ie left and right pages have mirror margins; blank pages inserted where needed):
\documentclass[a4paper,twoside]{ociamthesis}
% This one will format for one-sided binding (ie left margin > right margin; no extra blank pages):
%\documentclass[a4paper]{ociamthesis}
% This one will format for PDF output (ie equal margins, no extra blank pages):
%\documentclass[a4paper,nobind]{ociamthesis} 



%%%%% SELECT YOUR DRAFT OPTIONS
% Three options going on here; use in any combination.  But remember to turn the first two off before
% generating a PDF to send to the printer!

% This adds a "DRAFT" footer to every normal page.  (The first page of each chapter is not a "normal" page.)
\fancyfoot[C]{\emph{DRAFT Printed on \today}}  

% This highlights (in blue) corrections marked with (for words) \mccorrect{blah} or (for whole
% paragraphs) \begin{mccorrection} . . . \end{mccorrection}.  This can be useful for sending a PDF of
% your corrected thesis to your examiners for review.  Turn it off, and the blue disappears.
\correctionstrue


%%%%% BIBLIOGRAPHY SETUP
% Note that your bibliography will require some tweaking depending on your department, preferred format, etc.
% The options included below are just very basic "sciencey" and "humanitiesey" options to get started.
% If you've not used LaTeX before, I recommend reading a little about biblatex/biber and getting started with it.
% If you're already a LaTeX pro and are used to natbib or something, modify as necessary.
% Either way, you'll have to choose and configure an appropriate bibliography format...

% The science-type option: numerical in-text citation with references in order of appearance.
\usepackage[style=numeric-comp, sorting=none, backend=biber, doi=false, isbn=false]{biblatex}
\newcommand*{\bibtitle}{References}

% The humanities-type option: author-year in-text citation with an alphabetical works cited.
%\usepackage[style=authoryear, sorting=nyt, backend=biber, maxcitenames=2, useprefix, doi=false, isbn=false]{biblatex}
%\newcommand*{\bibtitle}{Works Cited}

% This makes the bibliography left-aligned (not 'justified') and slightly smaller font.
\renewcommand*{\bibfont}{\raggedright\small}

% Change this to the name of your .bib file (usually exported from a citation manager like Zotero or EndNote).
\addbibresource{references.bib}

% Uncomment this if you want equation numbers per section (2.3.12), instead of per chapter (2.18):
%\numberwithin{equation}{subsection}



%%%%% THESIS / TITLE PAGE INFORMATION
% Everybody needs to complete the following:
\title{Suitably impressive thesis title}
\author{Your Name}
\college{Your College}

% Master's candidates who require the alternate title page (with candidate number and word count)
% must also un-comment and complete the following three lines:
%\masterssubmissiontrue
%\candidateno{933516}
%\wordcount{28,815}

% Uncomment the following line if your degree also includes exams (eg most masters):
%\renewcommand{\submittedtext}{Submitted in partial completion of the}
% Your full degree name.  (But remember that DPhils aren't "in" anything.  They're just DPhils.)
\degree{Doctor of Philosophy}
% Term and year of submission, or date if your board requires (eg most masters)
\degreedate{Michaelmas 2014}


%%%%% YOUR OWN PERSONAL MACROS
% This is a good place to dump your own LaTeX macros as they come up.

% To make text superscripts shortcuts
	\renewcommand{\th}{\textsuperscript{th}} % ex: I won 4\th place
	\newcommand{\nd}{\textsuperscript{nd}}
	\renewcommand{\st}{\textsuperscript{st}}
	\newcommand{\rd}{\textsuperscript{rd}}

%%%%% THE ACTUAL DOCUMENT STARTS HERE
\begin{document}



%%%%% CHOOSE YOUR LINE SPACING HERE
% This is the official option.  Use it for your submission copy and library copy:
\setlength{\textbaselineskip}{22pt plus2pt}
% This is closer spacing (about 1.5-spaced) that you might prefer for your personal copies:
%\setlength{\textbaselineskip}{18pt plus2pt minus1pt}

% You can set the spacing here for the roman-numbered pages (acknowledgements, table of contents, etc.)
\setlength{\frontmatterbaselineskip}{17pt plus1pt minus1pt}

% Leave this line alone; it gets things started for the real document.
\setlength{\baselineskip}{\textbaselineskip}


%%%%% CHOOSE YOUR SECTION NUMBERING DEPTH HERE
% You have two choices.  First, how far down are sections numbered?  (Below that, they're named but
% don't get numbers.)  Second, what level of section appears in the table of contents?  These don't have
% to match: you can have numbered sections that don't show up in the ToC, or unnumbered sections that
% do.  Throughout, 0 = chapter; 1 = section; 2 = subsection; 3 = subsubsection, 4 = paragraph...

% The level that gets a number:
\setcounter{secnumdepth}{2}
% The level that shows up in the ToC:
\setcounter{tocdepth}{2}


%%%%% ABSTRACT SEPARATE
% This is used to create the separate, one-page abstract that you are required to hand into the Exam
% Schools.  You can comment it out to generate a PDF for printing or whatnot.
\begin{abstractseparate}
	Your abstract text goes here.  Check your departmental regulations, but generally this should be less than 300 words.  See the beginning of Chapter~\ref{ch:2-litreview} for more.
 % Create an abstract.tex file in the 'text' folder for your abstract.
\end{abstractseparate}


% JEM: Pages are roman numbered from here, though page numbers are invisible until ToC.  This is in
% keeping with most typesetting conventions.
\begin{romanpages}

% JEM: By default, this template uses the traditional Oxford "Belt Crest". Un-comment the following
% line to use the newer, "Blue Square" logo:
% \renewcommand{\crest}{{\includegraphics[width=4.2cm, height=4.2cm]{figures/newlogo.pdf}}}

% Title page is created here
\maketitle

%%%%% DEDICATION -- If you'd like one, un-comment the following.
%\begin{dedication}
%This thesis is dedicated to\\
%someone\\
%for some special reason\\
%\end{dedication}

%%%%% ACKNOWLEDGEMENTS -- Nothing to do here except comment out if you don't want it.
\begin{acknowledgements}
 	

\end{acknowledgements}

%%%%% ABSTRACT -- Nothing to do here except comment out if you don't want it.
\begin{abstract}
	Your abstract text goes here.  Check your departmental regulations, but generally this should be less than 300 words.  See the beginning of Chapter~\ref{ch:2-litreview} for more.

\end{abstract}

%%%%% MINI TABLES
% This lays the groundwork for per-chapter, mini tables of contents.  Comment the following line
% (and remove \minitoc from the chapter files) if you don't want this.  Un-comment either of the
% next two lines if you want a per-chapter list of figures or tables.
\dominitoc % include a mini table of contents
%\dominilof  % include a mini list of figures
%\dominilot  % include a mini list of tables

% This aligns the bottom of the text of each page.  It generally makes things look better.
\flushbottom

% This is where the whole-document ToC appears:
\tableofcontents

\listoffigures
	\mtcaddchapter
% \mtcaddchapter is needed when adding a non-chapter (but chapter-like) entity to avoid confusing minitoc

% Uncomment to generate a list of tables:
%\listoftables
%	\mtcaddchapter

%%%%% LIST OF ABBREVIATIONS
% This example includes a list of abbreviations.  Look at text/abbreviations.tex to see how that file is
% formatted.  The template can handle any kind of list though, so this might be a good place for a
% glossary, etc.
% First parameter can be changed eg to "Glossary" or something.
% Second parameter is the max length of bold terms.
\begin{mclistof}{List of Abbreviations}{3.2cm}

\item[1-D, 2-D] One- or two-dimensional, referring in this thesis to spatial dimensions in an image.


\end{mclistof} 


% The Roman pages, like the Roman Empire, must come to its inevitable close.
\end{romanpages}


%%%%% CHAPTERS
% Add or remove any chapters you'd like here, by file name (excluding '.tex'):
\flushbottom
\begin{savequote}[8cm]
\end{savequote}

\chapter*{\label{ch:1-intro}Introduction} 
\minitoc

The Deep Underground Neutrino Experiment (DUNE) \cite{DUNE:2020TDR1} will be a new generation accelerator oscillation neutrino experiment. It will include three main components: a powerful wide-band neutrino beam situated at Fermilab, a multi-instrumented Near Detector positioned a few hundred meters from the neutrino source and a Far Detector situated 1300 km away from the source in South Dakota. The key goal of the DUNE experiment will be to initiate a new era of precision in the field of neutrino flavour oscillations by performing a complete range of measurements using both neutrino and antineutrino beams. Particular emphasis will be put on the measurement of charge parity violation in neutrino oscillations and determining the order of the neutrino mass eigenstates.

In order to perform neutrino oscillation measurements, a prediction for the expected signal and background at the Far Detector as a function of the oscillation parameters, needs to be produced. The prediction is then compared with the measured flavor-tagged neutrino spectra at the Far Detector,
in order to produce estimates for the parameters that regulate the oscillation probability. Producing this prediction requires the determination of the neutrino flux at production, the neutrino interaction cross sections and the response of the detector: all of these factors are affected by systematic uncertainties that need to be constrained. The DUNE Near Detector has been designed to specifically
address each element \cite{DUNE:2021NDCDR}.

The DUNE ND will include as one of its main components a high pressure gas time projection chamber (HpGTPC) based on Argon called ND-GAr. The TPC technology has enjoyed ample success in high-energy particle physics. Since its original proposal by Nygren in 1975~\cite{NygrenTPC}, it has been utilized in various experiments and setups ~\cite{ATTIE200989,Hilke:2010zz}. One of the most notable examples of gas TPC's in modern particle physics is the ALICE experiment at CERN~\cite{ALICE:2008ngc}. The design of ND-GAr is heavily inspired by ALICE, to the point that the multi-wire proportional chambers that will be used by ND-GAr will be repurposed from the ALICE detector after its recent upgrade~\cite{ALICE:2023udb}. The use of a gas TPC in a neutrino experiment is unprecedented and it is motivated by the low tracking thresholds and extreme levels of precision achievable using this technology. These features will be particularly useful in the study of neutrino interactions and nuclear effects, which are one of the key sources of systematic uncertainty in the oscillation measurement.

In this thesis we present the conception, development and testing of what is now the standard momentum reconstruction algorithm used by the TPC component of the ND-GAr detector. The algorithm is based on the Kalman Filter technique, which is an iterative Bayesian technique often used in the field of particle tracking. This Kalman Filter application presents some novel features which allow it to reconstruct very long low energy tracks, which is particularly beneficial to a neutrino experiment such as ND-GAr. The method was developed in collaboration with experts of the ALICE collaboration and is envisioned to be used by both experiments. A similar more simple algorithm was also developed for ND-GAr-Lite, a now abandoned temporary muon spectrometer design centered on a multi-layer scintillator tracker,that would have substituted ND-GAr in the early days of the experiment's lifetime.

The impact of this novel Kalman Filter was tested in a significant physics application: as already mentioned, one of the key roles of the ND-GAr detector will be to act as a laboratory for the study of nuclear effects in neutrino interactions. One of the key techniques that will be used by the experiment is transverse kinematic imbalance (TKI)~\cite{Lu:2015hea, PhysRevC.94.015503, PhysRevC.99.055504, Cai:2019jzk, PhysRevD.102.033005}, a method inspired by the missing energy concept in high energy physics. It has been suggested that TKI could be used in a HpgTPC to isolate a sample of neutrino-Hydrogen interactions present in the gas mixture. Hydrogen interactions are devoid of nuclear effects and represents an ideal testing ground for the characterization of these crucial systematic uncertainty sources. The efficacy of the technique is in large part determined by the resolution of the detector, making the impact of the novel Kalman Filter very directly detectable.

This thesis is structured as follows:
\begin{itemize}
    \item Chapter \ref{ch:2-litreview} offers an introduction to neutrino physics and some key experimental particle physics concepts.
    \item Chapter \ref{ch:3-DUNE} provides a description of the DUNE experiment and its physics goals, with particular focus on the Near Detector and ND-GAr.
    \item Chapter \ref{ch:4-KF-NDGArLite} introduces the key concepts behind the Kalman Filter technique and offers a description of a Kalman Filter application developed for the ND-GAr-Lite temporary muon spectrometer.
    \item Chapter \ref{ch:5-KF-NDGArToy} introduces the novel Kalman Filter developed for the ND-GAr detector and investigates its performance.
    \item Chapter \ref{ch:6-TKI} describes a study on the capability of the ND-GAr detector of using transverse kinematic imbalance to isolate a sample of neutrino-Hydrogen interactions, exploring the impact of the new Kalman Filter algorithm.
\end{itemize}









\include{text/ch2-litreview}


%% APPENDICES %% 
% Starts lettered appendices, adds a heading in table of contents, and adds a
%    page that just says "Appendices" to signal the end of your main text.
\startappendices
% Add or remove any appendices you'd like here:
\begin{savequote}[8cm]
\end{savequote}

\chapter{\label{app:1-cardiophys}Extra Plots}

\minitoc

Appendices are just like chapters.  Their sections and subsections get numbered and included in the table of contents; figures and equations and tables added up, etc.  

\section{First Extra}
\label{sec:FirstApp}




%%%%% REFERENCES

% JEM: Quote for the top of references (just like a chapter quote if you're using them).  Comment to skip.
\begin{savequote}[8cm]
The first kind of intellectual and artistic personality belongs to the hedgehogs, the second to the foxes \dots
  \qauthor{--- Sir Isaiah Berlin \cite{berlin_hedgehog_2013}}
\end{savequote}

\setlength{\baselineskip}{0pt} % JEM: Single-space References

{\renewcommand*\MakeUppercase[1]{#1}%
\printbibliography[heading=bibintoc,title={\bibtitle}]}


\end{document}
