\begin{savequote}[8cm]
\end{savequote}

\chapter*{\label{ch:10-End}Conclusions}

The main goal of this thesis was to describe the development of a novel Kalman Filter algorithm for the ND-GAr detector and show its impact in achieving the detector's physics goals, specifically in the study of neutrino interactions and nuclear effects. 

In Ch. \ref{ch:4-KF-NDGArLite} we introduced the Kalman Filter technique and its application to particle tracking and we described the development and testing of a KF for the ND-GAr-Lite detector named \texttt{KF-Lite}. The ND-GAr-Lite detector was a proposed temporary muon spectrometer designed to substitute ND-GAr in the early days of DUNE experiment's data tacking. The testing of \texttt{KF-Lite} started with the production of a toy Monte Carlo Tool, which allowed to validate each component of the algorithm individually. Once the algorithm was demonstrated to be mature and internally consistent, it was applied to data simulated with \texttt{GArSoft}, the software tool developed by the ND-GAr collaboration. The performance of \texttt{KF-Lite} was tested on a sample of mono-energetic forward-going muons and was compared with the previously available reconstruction algorithm which consisted of an interative linear regression method \texttt{ILRM}. \texttt{KF-Lite} was shown to significantly outperform the \texttt{ILRM}, fully removing a $\sim4\%$ momentum bias present in the original reconstruction.

In Ch. \ref{ch:5-KF-NDGArToy} we described a novel Kalman Filter constructed for the ND-GAr detector and based on the parametrization and the infrastructure originally developed by the ALICE experiment. The new algorithm, which we named the Complete Kalman Filter or \texttt{CKF}, includes a novel feature which allows for the reconstruction of very long \enquote{looping} tracks produced by low energy light particles. The impact of this technique and the internal consistency of the algorithm was demonstrated in a toy Monte Carlo study using a wide range of detector and particle characteristics. In particular it was shown that the novel \enquote{loop-following} method has the potential of improving both the detection efficiency and the resolution significantly for light particles such as electrons, muons and pions. Once the algorithm was demonstrated to be fully mature, it was integrated in \texttt{GArSoft}. Two studies were produced which employed a sample of primary particles from $\nu_\mu$ CC interactions and a sample of forward-going muons respectively. In both cases it was shown that the \texttt{CKF} improves the performance of the ND-GAr detector. Direct comparison were done with the previous reconstruction algorithm, which we named the GArSoft Kalman Filter \texttt{GKF} , showing improvement in angular and momentum resolution and bias for all particle types and especially protons.

The impact of the \texttt{CKF} was demonstrated in Ch. \ref{ch:6-TKI}. In this final chapter we described a study designed to evaluate the feasibility of using transverse kinematic imbalance (TKI) techniques to isolate a sample of neutrino-Hydrogen interactions in the ND-GAr detector. The performance was evaluated in terms of double transverse kinematic imbalance resolution $\sigma(\delta p_\text{TT})$, which is the TKI variable used to separate the Hydrogen interactions from the rest. The study was performed on a sample of $\nu_\mu$ CC interactions simulated with \texttt{GArSoft} and showed that the new \texttt{CKF} algorithm was capable of producing a resolution of $\sigma(\delta p_\text{TT})=(11.9\pm0.7)$. This result represents an improvement of a factor of $\sim 2$ compared to the resolution found for \texttt{GKF}, which was is  $\sigma(\delta p_\text{TT})=(19.7\pm1.3)$. Finally the efficiency and purity for the Hydrogen sample cut on the nominal ND-GAr gas mixture were evaluated. The technique was shown to have good potential for ND-GAr, but different gas mixtures will need to be considered for it to be fully effective. 
