\begin{savequote}[8cm]
Ci sono soltanto due possibili conclusioni: \\
se il risultato conferma l'ipotesi,\\
allora hai appena fatto una misura. \\
Se il risultato è contrario alle ipotesi, \\
allora hai fatto una scoperta.

There are only two possible conclusions:\\
if the result confirms the hypothesis, \\
then you made a measurement.\\
If the result contradicts the hypothesis, \\
then you made a discovery.
  \qauthor{--- Enrico Fermi}
\end{savequote}

\chapter{\label{ch:2-litreview}Theoretical Background}
\minitoc
\section{Brief history of neutrinos}
\section{Neutrino in the Standard Model and neutrino masses}
\section{Theory of neutrino Oscillations}
\subsection{Two flavour scenario}
\subsection{Three flavour Oscillation in vacuum}
\subsection{Charge-Parity Simmetry Violation}
\subsection{Mass-Hierarchy}
\subsection{MSW effect}
\section{Neutrino Oscillation Experiments}
\subsection{Solar Experiments}
\subsection{Reactor Experiments}
\subsection{Atmospheric experiments}
\subsection{Accelerator Experiments}
\subsection{$\delta_\textrm{CP}$ experimental results}
\subsection{Mass hierarchy experimental results}
\section{Passage of particles through matter}
\subsection{Energy Loss}
\subsection{Multiple Scattering}
\section{Track Reconstruction}
\subsection{The Kalman Filter technique}
