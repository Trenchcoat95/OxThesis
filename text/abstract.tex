
The Deep Underground Neutrino Experiment (DUNE) is a next-generation neutrino oscillation experiment designed to achieve high precision in studying neutrino flavour oscillations. The experiment aims to measure charge-parity violation in neutrino oscillations and determine the neutrino mass hierarchy. DUNE consists of a wide-band neutrino beam at Fermilab, a Near Detector (ND) located a few hundred meters from the neutrino source, and a Far Detector 1300 km away in South Dakota. To perform neutrino oscillation measurements, predictions of the expected signal and background at the Far Detector are required, which depend on the ND to constrain systematic uncertainties, including neutrino flux and interaction cross sections. The ND features a high-pressure gas time projection chamber (HpGTPC) called ND-GAr, based on technology used in the ALICE experiment at CERN. The unprecedented use of this technology in neutrino physics allows for low tracking thresholds and high precision, which are critical for addressing systematic uncertainties arising from nuclear effects. A novel momentum reconstruction algorithm using a Kalman Filter was developed for ND-GAr, enabling the reconstruction of low-energy tracks. Its impact was demonstrated by studying the ND-GAr performance in the application of the transverse kinematic imbalance (TKI) technique. TKI variables can be used to isolate neutrino-Hydrogen interactions, providing a clean sample for systematic studies free of nuclear effects. The novel Kalman Filter is shown to enhance ND-GAr’s capability in the study of nuclear effects through TKI and in supporting DUNE’s scientific goals in general.
