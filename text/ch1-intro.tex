\begin{savequote}[8cm]
\end{savequote}

\chapter{\label{ch:1-intro}Introduction} 
\minitoc
The time projection chamber (TPC) has enjoyed ample success in high-energy particle physics. Since its original proposal by Nygren in 1975~\cite{NygrenTPC}, it has been utilized in various experiments and setups~\cite{ATTIE200989,Hilke:2010zz}. In a TPC, signal and track formation are achieved through the production of ionization electrons induced by the energy deposition of passing charged particles. The electrons then drift towards a sensor region in an electric field produced through an electrode plane. Subsequently, the electrons undergo multiplication through electromagnetic avalanches and are read out using technologies such as multi-wire proportional chambers (MWPCs)~\cite{Charpak:1968kd} or gas electron multipliers (GEMs)~\cite{SAULI1997531}. Additionally, the TPC is usually equipped with a magnetic field, enabling  momentum measurement by curvature and charge identification. The avalanche-induced signals provide hit coordinates in two dimensions, while the drift time provides the third.

The ALICE TPC at the LHC stands out as the most notable among those currently operational~\cite{ALICE:2008ngc}. ALICE is a nucleus-nucleus collision experiment, designed to study the physics of strongly interacting matter at extreme values of energy density and temperature. The gas TPC technology was chosen by the ALICE collaboration due to its robustness in providing charged-particle momentum measurements with good two-track separation, particle identification, and vertex determination, even at the extreme levels of occupancy reached in Pb-Pb collisions. A similar TPC, but relatively smaller, has been used by the STAR experiment at RHIC~\cite{STAR:2002eio}. Recently, ALICE has undergone a significant upgrade~\cite{ALICE:2023udb}, sparking renewed interest in TPC R\&D. 

The TPC technology is also heavily discussed in the realm of accelerator neutrino experiments, where it is typically used in the form of liquid argon TPCs~\cite{Rubbia:1977zz} as an interaction target and a tracking device, such as those employed in the Short-Baseline Neutrino program (SBN)~\cite{Machado:2019oxb} and as the Deep Underground Neutrino Experiment (DUNE) Far Detector~\cite{DUNE:2020TDR4}. Alternatively, gas TPCs like those of the T2K Near Detector~\cite{T2KND280TPC:2010nnd} serve as trackers for particles produced in neutrino interactions in the upstream denser components of the detector. DUNE will include the Gaseous Argon Near Detector (ND-GAr) in its near detector complex. ND-GAr will feature a high-pressure gas TPC (HPgTPC), heavily inspired by ALICE's design~\cite{DUNE:2021NDCDR}.

ND-GAr's TPC will have a cylindrical shape with the same dimensions of the ALICE TPC: a radius of 250 cm and a length of 500 cm. It will also incorporate the recently decommissioned MWPCs used by the ALICE experiment up to Run-3~\cite{Adolfsson_2021}, which achieved hit resolutions of approximately 1 mm~\cite{LIPPMANN2012}. However, ND-GAr will not feature an internal tracking system; instead, its central region will be filled with additional MWPCs, making it the largest gas TPC ever built. Furthermore, its gas mixture will be argon-based and maintained at a pressure of 10 atm, whereas ALICE operates at atmospheric pressure. ND-GAr's design is unique in that its TPC will have sufficient mass to provide its own sample of neutrino interactions while maintaining relatively low tracking thresholds. These characteristics will make it an ideal laboratory for studying neutrino interactions on gas, while also providing charge separation and full 4$\pi$ acceptance. ND-GAr's physics program will be centered on the reduction of systematic uncertainties in the neutrino oscillation measurement. A major source of systematics derives from the nuclear medium effects in neutrino interactions, which are less understood for heavier nuclei than carbon~\cite{Mosel:2016cwa}. ND-GAr has the potential to be extremely useful in the study of nuclear effects, using a variety of techniques, including transverse kinematic imbalance~\cite{Lu:2015hea, PhysRevC.94.015503, PhysRevC.99.055504, Cai:2019jzk, PhysRevD.102.033005}. The efficacy of these studies depend heavily on the detector's reconstruction resolution.

The Kalman Filter, an iterative Bayesian technique, facilitates estimating the state of a dynamic system by reconciling discrete measurements with predictions derived from prior knowledge of the system. Introduced by Kalman in 1960~\cite{Kalman_OG} and independently discovered by Stratonovich a year prior~\cite{stratonovich}, the technique has been the standard in TPC track fitting since its introduction by the DELPHI experiment~\cite{Kalman_app}, and remains the method with the best overall performance for most applications~\cite{RevModPhys.82.1419}. The Kalman Filter developed by the ALICE experiment for track formation and reconstruction can be considered the state of the art in the field~\cite{Ivanov:2003yr, Arslandok:2022dyb}, but it has some limitations which make its direct application to a neutrino experiment such as ND-GAr problematic.




