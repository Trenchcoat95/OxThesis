\begin{savequote}[8cm]
\end{savequote}

\chapter*{\label{ch:1-intro}Introduction} 
\minitoc

The Deep Underground Neutrino Experiment (DUNE) \cite{DUNE:2020TDR1} will be a new generation accelerator oscillation neutrino experiment. It will include three main components: a powerful wide-band neutrino beam situated at Fermilab, a multi-instrumented Near Detector positioned a few hundred meters from the neutrino source and a Far Detector situated 1300 km away from the source in South Dakota. The key goal of the DUNE experiment will be to initiate a new era of precision in the field of neutrino flavour oscillations by performing a complete range of measurements using both neutrino and antineutrino beams. Particular emphasis will be put on the measurement of charge parity violation in neutrino oscillations and determining the order of the neutrino mass eigenstates.

In order to perform neutrino oscillation measurements, a prediction for the expected signal and background at the Far Detector as a function of the oscillation parameters, needs to be produced. The prediction is then compared with the measured flavor-tagged neutrino spectra at the Far Detector,
in order to produce estimates for the parameters that regulate the oscillation probability. Producing this prediction requires the determination of the neutrino flux at production, the neutrino interaction cross sections and the response of the detector: all of these factors are affected by systematic uncertainties that need to be constrained. The DUNE Near Detector has been designed to specifically
address each element \cite{DUNE:2021NDCDR}.

The DUNE ND will include as one of its main components a high pressure gas time projection chamber (HpGTPC) based on Argon called ND-GAr. The TPC technology has enjoyed ample success in high-energy particle physics. Since its original proposal by Nygren in 1975~\cite{NygrenTPC}, it has been utilized in various experiments and setups ~\cite{ATTIE200989,Hilke:2010zz}. One of the most notable examples of gas TPC's in modern particle physics is the ALICE experiment at CERN~\cite{ALICE:2008ngc}. The design of ND-GAr is heavily inspired by ALICE, to the point that the multi-wire proportional chambers that will be used by ND-GAr will be repurposed from the ALICE detector after its recent upgrade~\cite{ALICE:2023udb}. The use of a gas TPC in a neutrino experiment is unprecedented and it is motivated by the low tracking thresholds and extreme levels of precision achievable using this technology. These features will be particularly useful in the study of neutrino interactions and nuclear effects, which are one of the key sources of systematic uncertainty in the oscillation measurement.

In this thesis we present the conception, development and testing of what is now the standard momentum reconstruction algorithm used by the TPC component of the ND-GAr detector. The algorithm is based on the Kalman Filter technique, which is an iterative Bayesian technique often used in the field of particle tracking. This Kalman Filter application presents some novel features which allow it to reconstruct very long low energy tracks, which is particularly beneficial to a neutrino experiment such as ND-GAr. The method was developed in collaboration with experts of the ALICE collaboration and is envisioned to be used by both experiments. A similar more simple algorithm was also developed for ND-GAr-Lite, a now abandoned temporary muon spectrometer design centered on a multi-layer scintillator tracker,that would have substituted ND-GAr in the early days of the experiment's lifetime.

The impact of this novel Kalman Filter was tested in a significant physics application: as already mentioned, one of the key roles of the ND-GAr detector will be to act as a laboratory for the study of nuclear effects in neutrino interactions. One of the key techniques that will be used by the experiment is transverse kinematic imbalance (TKI)~\cite{Lu:2015hea, PhysRevC.94.015503, PhysRevC.99.055504, Cai:2019jzk, PhysRevD.102.033005}, a method inspired by the missing energy concept in high energy physics. It has been suggested that TKI could be used in a HpgTPC to isolate a sample of neutrino-Hydrogen interactions present in the gas mixture. Hydrogen interactions are devoid of nuclear effects and represents an ideal testing ground for the characterization of these crucial systematic uncertainty sources. The efficacy of the technique is in large part determined by the resolution of the detector, making the impact of the novel Kalman Filter very directly detectable.

This thesis is structured as follows:
\begin{itemize}
    \item Chapter \ref{ch:2-litreview} offers an introduction to neutrino physics and some key experimental particle physics concepts.
    \item Chapter \ref{ch:3-DUNE} provides a description of the DUNE experiment and its physics goals, with particular focus on the Near Detector and ND-GAr.
    \item Chapter \ref{ch:4-KF-NDGArLite} introduces the key concepts behind the Kalman Filter technique and offers a description of a Kalman Filter application developed for the ND-GAr-Lite temporary muon spectrometer.
    \item Chapter \ref{ch:5-KF-NDGArToy} introduces the novel Kalman Filter developed for the ND-GAr detector and investigates its performance.
    \item Chapter \ref{ch:6-TKI} describes a study on the capability of the ND-GAr detector of using transverse kinematic imbalance to isolate a sample of neutrino-Hydrogen interactions, exploring the impact of the new Kalman Filter algorithm.
\end{itemize}








